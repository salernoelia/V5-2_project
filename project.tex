\documentclass{article}

\usepackage[english]{babel}
\usepackage[letterpaper,top=2cm,bottom=2cm,left=3cm,right=3cm,marginparwidth=1.75cm]{geometry}
\usepackage{amsmath}
\usepackage{amssymb}
\usepackage{graphicx}
\usepackage{float}
\usepackage{subcaption}
\usepackage[colorlinks=true, allcolors=blue]{hyperref}

\title{Rosenzweig–MacArthur Reaction–Diffusion Model}
\author{Ege Seçgin, Elia Salerno}

\begin{document}
\maketitle

\begin{abstract}
This report investigates the spatial dynamics of the Rosenzweig-MacArthur predator-prey model using numerical simulations. We discretize the coupled non-linear parabolic partial differential equations using the Finite Difference Method (FDM), specifically the Forward-Time Central-Space (FTCS) scheme. We analyze the stability conditions of the discrete system and explore the formation of Turing patterns, demonstrating how spatial diffusion coupled with non-linear reaction kinetics drives the system from a homogeneous steady state to complex spatial structures.
\end{abstract}

\section{The Mathematical Model}

\subsection{Model Description and History}
% task 2
The Rosenzweig-MacArthur model, proposed in 1963, extends the classical Lotka-Volterra system to address structural instability. It incorporates logistic self-limitation for the prey and a saturating Holling type-II functional response for predation. The reaction-diffusion extension analyzed here allows for the study of spatial phenomena such as invasion waves and the "paradox of enrichment" across heterogeneous landscapes. % [cite: 200-203]

\subsection{Governing Equations}
% task 1
The model is classified as a system of coupled, non-linear, parabolic partial differential equations (PDEs). The evolution of prey density ($u$) and predator density ($v$) is given by:

\begin{align}
\frac{\partial u}{\partial t} &= D_u \nabla^2 u + ru \left (1- \frac{u}{K} \right) - \frac{\alpha uv}{1 + hu} \label{eq:u} \\
\frac{\partial v}{\partial t} &= D_v \nabla^2 v + \beta \frac{\alpha uv}{1+hu}-mv \label{eq:v}
\end{align}

\noindent where $\nabla^2 = \partial_{xx} + \partial_{yy}$ represents the Laplacian operator in two spatial dimensions. % [cite: 204]

\subsection{Parameter Definitions}
% task 3d
The system parameters are defined as follows: % [cite: 206-213]
\begin{itemize}
    \item $u, v$: Prey and Predator population densities ($ind/m^2$).
    \item $D_u, D_v$: Diffusion coefficients ($m^2/d$).
    \item $r$: Prey intrinsic growth rate ($d^{-1}$).
    \item $K$: Prey carrying capacity ($ind/m^2$).
    \item $\alpha$: Attack rate ($m^2/(ind \cdot d)$).
    \item $h$: Handling time factor ($ind^{-1}$).
    \item $\beta$: Conversion efficiency (dimensionless).
    \item $m$: Predator mortality rate ($d^{-1}$).
\end{itemize}

\subsection{Assumptions and Numerical Choices}
% Task 3a, 3b, 3c
To numerically solve the system, we utilize the Forward-Time Central-Space (FTCS) scheme, an explicit method chosen for its ease of implementation and sufficient accuracy for observing qualitative spatial pattern formation, and apply Periodic Boundary Conditions, assuming the domain represents a small patch within a larger, continuous ecosystem to eliminate edge effects and mimic an infinite domain. We initialize the fields with a homogeneous steady state plus small random noise, which is essential to break symmetry because, without noise, the diffusion terms would remain zero and no spatial patterns would emerge from a uniform field. Finally, to observe Turing instabilities, we select diffusion coefficients such that $D_v \gg D_u$, meaning the inhibitor diffuses faster than the activator, deviating slightly from generic biological ranges to ensure mathematical instability.

\section{Discretization}
% task 4
We discretize the spatial domain into an $N \times N$ grid with step size $\Delta x$ and time into steps of $\Delta t$.

The Laplacian $\nabla^2 u$ is approximated using the five-point central difference stencil:
\begin{equation}
\nabla^2 u \approx \frac{u_{i+1,j} + u_{i-1,j} + u_{i,j+1} + u_{i,j-1} - 4u_{i,j}}{(\Delta x)^2}
\end{equation}

The time derivative is approximated using the forward Euler method:
\begin{equation}
\frac{\partial u}{\partial t} \approx \frac{u^{n+1}_{i,j} - u^n_{i,j}}{\Delta t}
\end{equation}

Substituting these into the governing equations yields the update rule for the prey (and similarly for the predator):
\begin{equation}
u^{n+1}_{i,j} = u^n_{i,j} + \Delta t \left[ D_u \frac{\Delta^2 u^n_{i,j}}{(\Delta x)^2} + R_u(u^n_{i,j}, v^n_{i,j}) \right]
\end{equation}
where $R_u$ represents the reaction terms.

\subsection{Truncation Error Analysis}
% task 5
The local truncation error of the FTCS scheme is derived from the Taylor series expansion.
\begin{itemize}
    \item \textbf{Temporal Error:} The forward Euler approximation introduces an error of order $O(\Delta t)$.
    \item \textbf{Spatial Error:} The central difference approximation for the second derivative introduces an error of order $O(\Delta x^2)$.
\end{itemize}
Thus, the total truncation error is $O(\Delta t, \Delta x^2)$. The method is first-order accurate in time and second-order accurate in space.

\section{Stability Analysis}
% task 7
For explicit diffusion schemes, stability is governed by the Courant-Friedrichs-Lewy (CFL) condition. For a 2D diffusion equation, the von Neumann stability analysis requires:
\begin{equation}
\Delta t \le \frac{(\Delta x)^2}{4 \max(D_u, D_v)}
\end{equation}
In our implementation, we calculate the maximum allowable $\Delta t$ dynamically based on the chosen diffusion coefficients and grid spacing. We use a safety factor (e.g., $0.9 \cdot \Delta t_{max}$) to ensure stability in the presence of the non-linear reaction terms. Experimental verification showed that exceeding this limit results in numerical overflow ("blow-up").

\section{Results and Discussion}

\subsection{Baseline Solution}
% task 6 & 9 
The simulation was run on a $100 \times 100$ grid. Starting from random noise, the system initially exhibits small fluctuations which grow over time due to the Turing instability mechanism. 

After sufficient time integration ($T=2000$), the system converges to a heterogeneous steady state. We observe distinct spatial patterns (spots or labyrinthine stripes depending on specific parameter tuning). The prey density ($U$) and predator density ($V$) fields are spatially correlated but inversely phased in certain regions, reflecting the predation dynamics.

\subsection{Parameter Sensitivity}
% task 8
We investigated the effect of the diffusion ratio $d = D_v / D_u$, observing that when $D_v \approx D_u$, the system stabilizes to a homogeneous uniform color, or no patterns, as the inhibitor cannot diffuse fast enough to contain the activator locally. As the ratio increases, $D_v \gg D_u$, spatial symmetry breaks, leading to the formation of stable Turing patterns, which confirms that differential diffusivity is a necessary condition for pattern formation in this reaction-diffusion system.

\section{Advanced Analysis (Optional)}

\subsection{Comparison of Schemes}
% task 10 (Optional)
While FTCS is sufficient for qualitative pattern observation, it imposes a strict time-step restriction. An alternative would be the Crank-Nicolson scheme (implicit), which is unconditionally stable for the diffusion term, allowing for larger $\Delta t$ at the cost of solving a linear system at each step. For this project, the computational cost of the explicit scheme was negligible, making FTCS the preferred choice.

\bibliographystyle{alpha}
\bibliography{bibliography}

\end{document}